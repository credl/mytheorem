\documentclass{article}

\usepackage[outsourcebydefault]{mytheorem}						% use this command to outsource theorem proofs by default
%\usepackage{mytheorem}											% use this command to inline theorem proofs by default

%\renewcommand{\proofsecref}{Section~\ref{sec:arbitrarysec}}	% can be used to redefine the text that refers to the proof section when theorems are exceptionally outsourced

% Declare theorem types here:
\newMyTheoremType{theorem}{Theorem}
\newMyTheoremType{lemma}{Lemma}

% Now the following commands are available for declaring theorems (analogously for lemmata and any other types declared with \newMyTheoremType)
% 
% addTheoremInline/addProofInline:
% 		Allows for adding theorems whose proofs are always inline, independent of the default setting.
% 
% addTheoremOutsourcedOnly/addProofOutsourcedOnly:
% 		Allows for adding theorems which appear only in the outsourced section (also their proofs), independent of default setting.
% 		In the text, not even the theorem is stated.
% 		This is intended to be used for lemmata which are only relevant in the proofs of other theorems and should therefore not occur in the text.
% 
% addTheoremOutsourced/addProofOutsourced:
% 		Allows for adding theorems whose proofs are always outsourced, independent of default setting.
% 
% addTheorem/addProof:
% 		Allows for placing proofs of theorems flexibly either inline or outsourced, depending on the setting of default setting.

\title{mytheorem.sty}
\author{Christoph Redl}

\begin{document}

	\maketitle
	
	\section{Introduction}
	
		This package allows for flexibly positioning theorem proofs
		either in the text or in a separate proof section in the appendix.

		To register a type of a theorem (e.g.~for lemmata, propositions, theorems, examples, etc),
		use the command
		\begin{center}
			\textbackslash{}newMyTheoremType\{[t]\}\{[T]\}
		\end{center}
		where \emph{[t]} is to be replaced by the type of theorem from the amsthm.sty package
		(e.g.~\emph{lemma}, \emph{theorem}, etc), and \emph{[T]} by the desired human-readable theorem name (e.g.~\emph{Lemma}, \emph{Theorem}, etc).
		This will then define the commands \textbackslash{}add[T], \textbackslash{}add[T]Inline, \textbackslash{}add[T]Outsourced, \textbackslash{}add[T]OutsourcedOnly,
		which can be used to state theorems as described next.
		
		The command \textbackslash{}addProof is always available and is used to define proofs for previously stated theorems (of any kind).

		For customizing theorem styles, use the commands \textbackslash{}newMyTheoremStyle and \textbackslash{}myTheoremStyle
		instead of \textbackslash{}newtheoremstyle and \textbackslash{}theoremstyle. They expect exactly the same parameters as the original methods
		from the amsthm.sty package.
		
	\section{Supported Types of Proofs}

		The package supports four types of proof positioning:
		\begin{enumerate}
			\item Theorems in the text with \textbf{flexibly positioned proofs}. All such proofs can easily be included in the text or outsourced by changing a single flag for the whole document.

				Commands are \\
				\textbackslash{}addTheorem (or \textbackslash{}addLemma, etc) and \\
				\textbackslash{}addProof.
			\item Theorems in the text with proofs which are \textbf{always inline} (e.g. for short proofs).

				Commands are \\
				\textbackslash{}addTheoremInline (or \textbackslash{}addLemmaInline, etc) and \\
				\textbackslash{}addProofInline.
			\item Theorems in the text with proofs which are \textbf{always outsourced} (e.g. very long proofs).

				Commands are \\
				\textbackslash{}addTheoremOutsourced (or \textbackslash{}addLemmaOutsourced, etc) and \\
				\textbackslash{}addProofOutsourcedOnly.
			\item Theorems which are, together with their proofs, \textbf{only stated in the appendix} (e.g. lemmata).

				Commands are \\
				\textbackslash{}addTheoremOutsourcedOnly (or \textbackslash{}addLemmaOutsourcedOnly, etc) and \\
				\textbackslash{}addProofOutsourced.
		\end{enumerate}
		
		Note that items 1, 2 and 3 will always state the theorem in the text but only the positioning of the proof differs,
		while item 4 concerns both the proof and the theorem.
		In particular, for item 3 only the proof will be outsourced (independent of the default setting),
		while for item 4 also the theorem will appear in the appendix only (also independent of the default setting).
		Item 3 is intended to be used when the theorem is relevant in the text but the proof is not or very long,
		while item 4 should be used when also the theorem is only relevant in the proof section (e.g.~because it is a lemma which is used in other proofs).

	\section{Examples}

		Some examples follow.
		It is suggested to compile the document twice using either the option from line 3 or 4 to see how the setting influences
		the positioning of the proofs.

		\addTheorem[Flexible Theorem Proof]{thm:1}{
			This example refers to item 1 in the list.
			The theorem is always stated in the text.
			The proof of this theorem appears inline or outsourced, depending on the default setting.
			If it is outsourced, then the theorem itself is stated both inline and before the outsourced proof.
		}

		\addProof{thm:1}{
			Proof of Theorem 1 (outsourced or inline depending on the settings).
		}

		\addTheoremInline[Inline Theorem Proof]{thm:2}{
			This example refers to item 2 in the list.
			The theorem is always stated in the text.
			The proof of this theorem always appears inline, independent of the default setting.
		}

		\addProofInline{thm:2}{
			Proof of Theorem 2 (always inline).
		}

		\addLemmaOutsourcedOnly[Theorem which appears only in the outsourced section]{lemma:1}{
			This example refers to item 4 in the list.
			This lemma \emph{and} its proof always \emph{both} appear outsourced, independent of the default setting.
			It is only relevant in the outsourced section, e.g., for proving Theorem~\ref{thm:3}.
		}

		\addProofOutsourcedOnly{lemma:1}{
			Proof of Lemma 1 (theorem and proof always outsourced).
		}

		\appendToProofs{
			This sentence is added to the proof section and allows for adding explanations of theorems which are only relevant in the proof section
			(e.g. in proofs of other theorems).
		}

		\addTheoremOutsourced[Outsourced Theorem Proof]{thm:3}{
			This example refers to item 3 in the list.
			The theorem is always stated in the text.
			The proof of this theorem always appears outsourced, independent of the default setting.
			If the default setting is inline (i.e., proofs are \emph{usually} inline), then it is explicitly noted in the text that this proof is exceptionally outsourced.
			If the default setting is outsourced (i.e., proofs are outsourced anyway), then this it not noted.
		}

		\addProofOutsourced{thm:3}{
			Proof of Theorem 3 (always outsourced) using Lemma~\ref{lemma:1}.
		}

	\ifHaveProofs{
		\appendix
		\section{Proof Appendix}
		\label{sec:proofs}

			The outsourced proofs (and the repetition of the according theorems) will follow here:

			\proofs{}
	}

\end{document}
